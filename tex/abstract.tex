Distance education have been consolidating in Brazil gradatively as a learning modalitie at all stages of schooling, provided by learning management systems (LMS). This scenario have been demanding new strategies on pedagogical approaches that are capable of granted innovative teaching and learning. Aditionaly, there is also a concern with the practice of learning evaluation in these environments, since its by the evaluation activity that its expected to accompany the formal results of the teaching and learning process. In this sense, the following work seeks to investigate, through documentary research, which instruments and types of evaluation there are being used in these environments. Aspects such as frequency, formalization, function and object of evaluation will be captured quantitatively. As well as which features were most used. And if any relationship between these two approaches is being considered. An assertive discussion is presented on the results of this research, at the end of this document.
%É uma língua
%diferente e o texto deveria ser escrito de acordo com suas nuances (aproveite para ler
%\url{http://dx.doi.org/10.6061%2Fclinics%2F2014(03)01}).
%Por exemplo: \emph{This work presents useful information on how to create a scientific text to %describe
%and provide examples of how to use the Computer Science Department's \LaTeX\ class. The \unbcic\
%class defines a standard format for texts, simplifying the process of generating
%CIC documents and enabling authors to focus only on content. The standard was approved
%by the Department's professors and used to create this document. Future work includes
%continued support for the class and improvements on the explanatory text.}