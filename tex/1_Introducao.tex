

\label{chap:intro}
Conforme Alves~\cite{alves2009}, há registros que apontam que o ensino a distância já era realizado por professores particulares no Brasil, por meio de correspondência desde o final do século XIX. Sendo, mais tarde, também oferecido através de canais unidirecionais como programas educacionais de rádio e televisão. Nos dias atuais, no entanto, a popularidade dessa metodologia de ensino está diretamente relacionada à ampla propagação da infraestrutura da rede mundial de computadores (\emph{Internet}). Com computadores portáteis cada vez mais acessíveis e a comercialização do acesso à rede, avanços na legislação do país foram possíveis já em meados da década de 90. 

Por meio da promulgação da \acrfull{LDB}, por  meio  da Lei Federal nº. 9.394, em 20 de dezembro de 1996~\cite{brasilLDB}, o Brasil iniciou um período de expansão no apoio jurídico da modalidade de Educação a Distância (EAD), legitimando investimentos por parte das instituições de ensino e o desenvolvimento do setor. O Título VIII - Das Disposições Gerais, o artigo nº. 80 nos apresenta a seguinte diretiva:

\begin{quote}
[...]\\Art. 80. O Poder Público incentivará o desenvolvimento e a veiculação de programas de ensino a distância, em todos os níveis e modalidades de ensino, e de educação continuada.~\cite{brasilLDB}
\end{quote}%

Além de assegurar o amparo legal da \acrfull{EAD} para alunos de todos os níveis de ensino, a lei ainda prevê expressamente diversos estímulos para o uso de recursos e tecnologias desse modelo de ensino. Dentre eles, a criação de programas de capacitação para professores através de formação continuada a distância; a redução de custos sobre canais e meios de comunicação, com o intuito de incentivar a implantação das tecnologias necessárias ao seu uso; e a legalização de sua aplicação na complementação curricular da Educação Básica. 

Segundo dados do Censo da Educação Superior 2016~\cite{inep2016sinopse}, divulgado pelo \acrfull{Inep}, o Brasil conta atualmente com uma oferta de 1.662 cursos de graduação na modalidade de EAD, sendo oferecidos por 218 Instituições de Ensino Superior credenciadas pelo \acrfull{MEC}. Para se ter uma ideia mais clara do que isso representa, basta compararmos a taxa de crescimento de matrículas naquele ano. Enquanto o ensino presencial registrou uma queda de 0,08{\%} no número de novas matrículas, o ensino a distância registrou um crescimento de 7,2{\%}~\cite{exame@ead}.

Paralelamente, o Censo EAD Brasil 2016~\cite{abed2016censo}, conduzido pela \acrfull{ABED}, vem registrando um aumento na oferta de cursos regulamentados nas modalidades de \acrshort{EAD} e semipresencial. Somente no ensino técnico profissionalizante e na Educação Básica, o Censo mapeou um total de 532 cursos oferecidos por instituições credenciadas pelos \acrfull{CEEs}. Ainda, segundo dados analisados pela equipe do censo, os números revelaram que:

\begin{quote}
[...] 46{\%} das instituições que oferecem cursos regulamentados totalmente a distância tiveram um aumento no número de matrículas, mas somente 16{\%} delas observaram um aumento na sua rentabilidade. Mesmo assim, 37{\%} delas aumentaram os investimentos em 2016, e 31{\%} pretendem aumentar investimentos em cursos regulamentados totalmente a distância em 2017.
\end{quote}

Mas, talvez, o ponto mais significativo desvendado por esse estudo seja o resultado apresentado sobre a investigação do perfil dessas instituições, no qual se revela uma mudança do que são considerados como os principais desafios e obstáculos enfrentados por elas. O relatório aponta que, de 2010 até 2014, a evasão ocupava o topo das preocupações das entidades participantes do censo, perdendo em 2015 o posto para questões ligadas à inovação tecnológica e de infraestrutura. E, em 2016, o ponto de atenção principal, de acordo com as instituições, passou a ser o desafio de inovação em abordagens pedagógicas.

Além disso, em questões propostas pelo Censo, as instituições demonstraram um forte interesse em metodologias ativas, o que se confirmou por uma alto índice de uso de recursos interativos nas atividades propostas. Porém, ao analisar as relações de \textit{feedback}\footnote{Informações e recomendações fornecidas
ao aluno (pelo professor ou por seus pares) sobre o seu desempenho,
baseadas nos resultados de sua avaliação, as quais são planejadas para ajudar o aluno a melhorar seu desempenho.~\cite{grego@2013}} percebeu-se um predomínio da interação ``professor-aluno'', e de registro de notas dadas por disciplina. O que levou seus autores a concluir que
``a mudança nas tecnologias, portanto, teria vindo antes da mudança nas metodologias, ao menos do ponto de vista da avaliação''.

A preocupação com abordagens pedagógicas diferenciadas revela, acima de tudo, um amadurecimento dessas instituições e da própria modalidade de ensino. E diante do cenário descrito, percebe-se que os novos desafios redirecionam questionamentos a prática de avaliação para as aprendizagens que, na Educação a Distância, é conduzida através de um \acrfull{AVA} e de seus recursos tecnológicos. 

A motivação desse trabalho surge dessa mudança de enfoque percebida pelo setor educacional. Ou seja, no entendimento de que um AVA precisa, para manter um aluno engajado em aprender, de bons mecanismos de interação entre professor, tutor e aluno. Sendo capaz de fornecer, constantemente, informações de qualidade aos envolvidos no processo de aprendizagem, na intenção de que esses possam atuar no aperfeiçoamento da experiência de uso dos referidos ambientes. 

Alinhado ao que foi exposto, este trabalho pretende destacar a importância que a prática de uma avaliação para as aprendizagens adquire na modalidade de Educação a Distância. Em geral, é pela avaliação que se espera acompanhar os resultados formais do processo de ensino e aprendizagem. Os autores utilizados como base desse estudo, apontam a avaliação como uma ferramenta motivadora, capaz de diagnosticar e conduzir para a aprendizagem. Estudos sobre metodologias, ferramentas de apoio e funções da avaliação estão no cerne daquilo que será apresentado ao longo deste trabalho. Espera-se que esta proposta possa contribuir com o debate sobre o aprimoramento da Educação a Distância, através da discussão assertiva sobre o papel da avaliação e suas possibilidades de aplicação em Ambientes Virtuais de Aprendizagem. 

A seguir, é apresentado o delineamento do objeto de estudo, por meio dos trabalhos que serviram de referência para o seu enquadramento. Tanto a formulação do problema, quanto a hipótese de pesquisa são enunciados logo em seguida. Acompanhados pela descrição dos objetivos e do percurso metodológico. E, concluindo esta seção, é apresentado o detalhamento dos capítulos.
%%%%%%%%%%%%%%%%%%%%%%%%%%%%%%%%%%%%%%%%%%%%%%%%%%%%%%%%%%%%%%%%%%%%%%%%%%%%%%%%
%%%%%%%%%%%%%%%%%%%%%%%%%%%%%%%%%%%%%%%%%%%%%%%%%%%%%%%%%%%%%%%%%%%%%%%%%%%%%%%%
%%%%%%%%%%%%%%%%%%%%%%%%%%%%%%%%%%%%%%%%%%%%%%%%%%%%%%%%%%%%%%%%%%%%%%%%%%%%%%%%

\section{Tema delimitado por suas bases teóricas}%

    Tomando como base a motivação apresentada na introdução, esse trabalho buscou estudar os recursos tecnológicos disponíveis em Ambientes Virtuais de Aprendizagem sob o aspecto da avaliação para as aprendizagens. Para tanto, subsidiaram o presente trabalho:

\begin{enumerate}
	\item Os estudos e proposições sobre avaliação de Luckesi~\cite{luckesi2014avaliaccao} e os estudos de Bloom~\cite{bloom1983manual} sobre as funções da avaliação diagnóstica, formativa e somativa (classificatória), além de outas fontes e autores que ajudaram a estabelecer, dentro do contexto pedagógico, o papel da avaliação considerado neste trabalho.
	\item O levantamento dos aspectos funcionais realizado nas plataformas Moodle e BlackBoard, apresentado no Capítulo~\ref{chap:ava}, que possibilitou a catalogação das ferramentas de aprendizagem e de avaliação disponíveis nesses ambientes.
\end{enumerate}%
 
\bigskip
%%%%%%%%%%%%%%%%%%%%%%%%%%%%%%%%%%%%%%%%%%%%%%%%%%%%%%%%%%%%%%%%%%%%%%%%%%%%%%%%
%%%%%%%%%%%%%%%%%%%%%%%%%%%%%%%%%%%%%%%%%%%%%%%%%%%%%%%%%%%%%%%%%%%%%%%%%%%%%%%%
%%%%%%%%%%%%%%%%%%%%%%%%%%%%%%%%%%%%%%%%%%%%%%%%%%%%%%%%%%%%%%%%%%%%%%%%%%%%%%%%

\section{Pressuposto do Problema}%

Ao definir a avaliação da aprendizagem, Luckesi~\cite{luckesi2014avaliaccao} se adianta em diferenciar avaliação de julgamento. Segundo ele, o julgamento é ``um ato que distingue o certo do errado, incluindo o primeiro e excluindo o segundo''. A avaliação, por sua vez, é vista por ele como o ato de acolher uma situação ``tendo em vista dar-lhe suporte de mudança, se necessário''. Para ele, a avaliação, quando aplicada para redirecionar o erro para o acerto, pode servir como ``um mecanismo subsidiário do planejamento e da execução", capaz de articular o ensino com a aprendizagem.

Seguindo essa perspectiva, a questão que se deseja investigar é a de como um AVA viabiliza a prática dessa avaliação entendida como formativa.
%%%%%%%%%%%%%%%%%%%%%%%%%%%%%%%%%%%%%%%%%%%%%%%%%%%%%%%%%%%%%%%%%%%%%%%%%%%%%%%%
%%%%%%%%%%%%%%%%%%%%%%%%%%%%%%%%%%%%%%%%%%%%%%%%%%%%%%%%%%%%%%%%%%%%%%%%%%%%%%%%
%%%%%%%%%%%%%%%%%%%%%%%%%%%%%%%%%%%%%%%%%%%%%%%%%%%%%%%%%%%%%%%%%%%%%%%%%%%%%%%%

\section{Hipótese de Pesquisa}%

Pode-se elencar e categorizar os recursos oferecidos por um determinado AVA, em função dos tipos de avaliação, com a finalidade de mapear as metodologias de avaliações possíveis de se praticar nele. 

%\textbf{--> IMPORTANTE validar esta hipótese com a profa. Maria Emília. Achei estranho "métodos de técnicas de avaliações...".}
%%%%%%%%%%%%%%%%%%%%%%%%%%%%%%%%%%%%%%%%%%%%%%%%%%%%%%%%%%%%%%%%%%%%%%%%%%%%%%%%
%%%%%%%%%%%%%%%%%%%%%%%%%%%%%%%%%%%%%%%%%%%%%%%%%%%%%%%%%%%%%%%%%%%%%%%%%%%%%%%%
%%%%%%%%%%%%%%%%%%%%%%%%%%%%%%%%%%%%%%%%%%%%%%%%%%%%%%%%%%%%%%%%%%%%%%%%%%%%%%%%

\section{Objetivo Geral}%

Investigar a avaliação para as aprendizagens em AVA, por intermédio das funcionalidades disponíveis. 

%\textbf{--> RECOMENDO explicitar que esta classificação baseia-se em um modelo já existente.}

%%%%%%%%%%%%%%%%%%%%%%%%%%%%%%%%%%%%%%%%%%%%%%%%%%%%%%%%%%%%%%%%%%%%%%%%%%%%%%%%
%%%%%%%%%%%%%%%%%%%%%%%%%%%%%%%%%%%%%%%%%%%%%%%%%%%%%%%%%%%%%%%%%%%%%%%%%%%%%%%%
%%%%%%%%%%%%%%%%%%%%%%%%%%%%%%%%%%%%%%%%%%%%%%%%%%%%%%%%%%%%%%%%%%%%%%%%%%%%%%%%

\subsection{Objetivos Específicos}% 

\begin{itemize}
	\item Identificar, no plano teórico e social, quais são os tipos de avaliação relacionadas ao processo de ensino e aprendizagem.
	\item Catalogar as funcionalidades disponíveis em ambientes AVA pelo ângulo analítico-tecnológico, em busca de instrumentos de avaliação.
	\item Propor um mapeamento dos instrumentos de avaliação encontrados, com os tipos de avaliação identificadas.
	\item Realizar um estudo de caso com base em uma pesquisa documental . 
	%\textbf{--> NÃO ENTENDI muito bem o que seria esse objetivo}
%	\item Analisar os resultados obtidos e apresentar as considerações finais. \textbf{--> ANALISAR com qual finalidade? Precisas ser mais específica...}
\end{itemize}

%%%%%%%%%%%%%%%%%%%%%%%%%%%%%%%%%%%%%%%%%%%%%%%%%%%%%%%%%%%%%%%%%%%%%%%%%%%%%%%%
%%%%%%%%%%%%%%%%%%%%%%%%%%%%%%%%%%%%%%%%%%%%%%%%%%%%%%%%%%%%%%%%%%%%%%%%%%%%%%%%
%%%%%%%%%%%%%%%%%%%%%%%%%%%%%%%%%%%%%%%%%%%%%%%%%%%%%%%%%%%%%%%%%%%%%%%%%%%%%%%%

\section{Metodologia}%
No \refCap{chap:ref} foi realizada uma pesquisa exploratória, em referenciais bibliográficos, para identificar os tipos de avaliações possíveis em AVA.

A catalogação das funcionalidades disponíveis em AVA, no \refCap{chap:ava}, partiu de uma pesquisa exploratória em diretórios de dados de EAD para identificar quais ambientes poderiam servir de base para a identificação das funcionalidades. Após essa etapa, foi realizada a leitura dos catálogos de divulgação desses ambientes, em conjunto com técnicas de engenharia reversa em tela, para listar e descrever as funcionalidades.

Para estabelecer a relação entre as funcionalidades de avaliação e os tipos de avaliação, no \refCap{chap:map}, foi realizada uma inspeção da aplicabilidade entre elas por meio de respostas parametrizadas. Permitindo o mapeamento em uma matriz rastreável.  

O \refCap{chap:insp} faz uso de um estudo de caso documental no repositório de trabalhos acadêmicos da \acrfull{UnB}, e em uma amostragem da revisão sistemática sobre avaliação em AVA~\cite{Ferreira@2016}, apresentada no \acrfull{SBIE}, em 2016.

Por fim, o \refCap{chap:results} apresenta os resultados quantitativos das variáveis ``adoção de ferramentas'' e ``tipos de avaliação'', e compara os resultados com o matriz relacional de ferramentas x avaliação. 
%%%%%%%%%%%%%%%%%%%%%%%%%%%%%%%%%%%%%%%%%%%%%%%%%%%%%%%%%%%%%%%%%%%%%%%%%%%%%%%%
%%%%%%%%%%%%%%%%%%%%%%%%%%%%%%%%%%%%%%%%%%%%%%%%%%%%%%%%%%%%%%%%%%%%%%%%%%%%%%%%
%%%%%%%%%%%%%%%%%%%%%%%%%%%%%%%%%%%%%%%%%%%%%%%%%%%%%%%%%%%%%%%%%%%%%%%%%%%%%%%%

\section{A Composição dos Capítulos}\label{chap:comp}%

Tendo em vista uma melhor organização das temáticas que envolvem esse estudo, a monografia foi estruturada da seguinte forma:

\begin{description}
\item[Capítulo~\ref{chap:ref}:] Reúne a síntese dos estudos que serviram de referencial teórico para fundamentar o entendimento sobre as diversas formas de avaliação no âmbito educacional.
\item[Capítulo~\ref{chap:ava}:] Apresenta o \acrfull{AVA}, e realiza a catalogação de suas funcionalidades de aprendizagem e de avaliação.
\item[Capítulo 4:] Propõe um relacionamento entre as funcionalidades de avaliação presentes em um AVA com os diversos tipos de avaliação.
\item[Capítulo 5:] Realiza um estudo de caso documental em modelos de avaliação propostos para AVA, com o auxílio do levantamento dos tipos de avaliação, do catálogo de ferramentas e do mapeamento realizado.
%\textbf{ --> ME PARECE (pela explicação) que deveria ser o capítulo 4. O que achas?}
\item[Capítulo 6:] Apresenta uma discussão expositiva dos resultados obtidos, faz um relato das dificuldades encontradas e sugere aperfeiçoamentos.
\item[Capítulo 7:] Aponta as considerações finais e oportunidades de melhoria em trabalhos futuros. 
\end{description}

