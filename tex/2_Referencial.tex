\label{chap:ref}
Para entender claramente a importância da avaliação dentro do contexto educacional e, subsequentemente no âmbito da Educação a Distância, é necessário compreender o papel reformador que ela pode exercer em todo o contexto social. Uma forma de se fazer isso é examinar brevemente como se deu o processo de formação do modelo da educação escolar e como isso influenciou a atividade de avaliação no decorrer do seu percurso histórico. 

%\textbf{--> REVISAR OS VERBOS DO TEXTO. Em alguns momentos usas terceira pessoal e, em outras, a forma impessoal...}

Essas questões serão tratadas ao longo deste capítulo e, também, dos aspectos pelos quais a avaliação da aprendizagem é categorizada e caracterizada por pesquisadores e autores que estudam o tema. Esta fundamentação é importante para a articulação que será feita no decorrer do presente trabalho.%

\section{Provas para reprovar?}%
 Luckesi~\cite{luckesi2014avaliaccao} afirma que, ao longo da história, a prática escolar foi realizada dentro de um modelo teórico social que ``pressupôs a educação como um mecanismo de conservação e de reprodução da sociedade'', sistematizado para assegurar sua estabilidade e continuidade. Esse modelo de educação, conforme o autor explica, pôs o exercício da avaliação escolar a serviço da manutenção desse mecanismo de conservação, moldando a avaliação para se comportar como um ``instrumento disciplinador não só das condutas cognitivas como também das sociai''. Condição na qual se difundiu amplamente o que ele sintetizou como ``pedagogia de exames'', cujo objetivo geral era aprovar os resultados esperados, gerando, em suas próprias palavras, ``um deslocamento da ideia de captura de conteúdo e de aprendizagem para o eixo nota-aprovação''.

Conforme Luckesi~\cite{luckesi2014avaliaccao}, por muitos anos a confiança do sistema educacional esteve centrada nesse modelo de avaliação, que se consolidou em uma espécie de ``seletividade social''. Ao usar provas para reprovar a inconformidade (respostas erradas), o autor destaca que esse modelo inibiu o questionamento, desenvolvendo personalidades submissas ao ``sistema de exames''. Com base nesse entendimento, percebe-se que, embora exista uma certa conveniência em se usar a educação para o repasse da cultura e do conhecimento, conservando os saberes geração após geração, existe também um grande desafio que é o de garantir que a educação não seja reduzida meramente ao papel de doutrinação. A busca de novos rumos para a prática da avaliação, que se afastem desse modelo tradicional, descrito até aqui. 

\section{Avaliação para as aprendizagens}%

Contrário ao que foi exposto na seção anterior, a educação emancipadora é descrita por Luckesi~\cite{luckesi2014avaliaccao} como aquela que busca desenvolver o pensamento crítico e o constante aperfeiçoamento do entendimento que temos do mundo que nos cerca. De maneira tal que o modelo educacional possa ser concebido mais como um ``mecanismo de transformação social'', do que como um simples mecanismo de conservação e domesticação dos educandos.

Em defesa da construção desse mecanismo ``reformador'', Luckesi~\cite{luckesi2014avaliaccao} convida os docentes a realizar uma estratégia de avaliação concebida para servir de ``verificação do aprendido'', com o objetivo claro e definido de ``conduzir para o aprendizado''. Ainda, segundo o autor:

\begin{quote}
A avaliação poderá ser caracterizada como um forma de ajuizamento da qualidade do objeto avaliado, fator que implica uma tomada de posição a respeito do mesmo, para aceitá-lo ou transformá-lo. ~\cite{luckesi1978}
\end{quote}%

Considerando, então, que a intenção do processo educacional seria a de guiar para a aprendizagem, não há dificuldades em se constatar que a avaliação com a finalidade de aprovar/reprovar pouco auxilia na transformação do resultado obtido. Para que a avaliação seja capaz de permitir uma tomada de decisão, ela precisaria assumir uma postura que o autor classifica como ``diagnóstica''. Tal que seus resultados servissem ao propósito de remediar as falhas ao invés de apenas aceitá-las e convertê-las em notas e conceitos. Em outras palavras, a avaliação deveria ser usada como meio de apuração e de ``reorientação'' do ensino e da aprendizagem, nunca como um fim em si.  
 
Seguindo o posicionamento defendido pelo autor, a avaliação disgnóstica deve propiciar além da autocompreensão do aluno, fornecendo \textit{feedback} de seu nível de conhecimento, uma resposta ao professor em relação ao quanto seu trabalho está sendo efetivo. Isto é, deve servir para aprimorar o ensino e melhorar a aprendizagem. Ou ainda, nas palavras de Grego~\cite{grego@2013}, deve ajudar ``o aluno a aprender e o professor a ensinar''.

O trabalho aqui apresentado segue esse modo de pensar a avaliação, voltado para as aprendizagens e a constante aquisição de conhecimento pelo aprendiz. E capaz de estabelecer um curso de ação na busca do desenvolvimento de todos os envolvidos no processo de ensino.

\section{Aspectos e tipos da avaliação}%

A partir dessa apresentação de uma compreensão mais abrangente sobre o propósito construtivo da avaliação, esta seção inicia o exame dos \textbf{aspectos} relacionados à classificação da avaliação (composição, frequência, funções, dentre outros). E tenta identificar, nas literaturas de referência, os \textbf{tipos de avaliação} dentro de cada um desses aspectos.

\subsection{Quanto ao formalismo}%
Em relação à formalização, a  avaliação se divide em dois planos distintos, podendo ser formalizada de modo que o educando esteja ciente de estar sendo avaliado e seja capaz de entender os seus critérios; ou informal, ignorada pelo educando, e até mesmo inconsciente por parte do docente. Freitas~\cite{de2003ciclos} afirma que:

\begin{quote}
No plano da \textbf{avaliação formal} estão as técnicas e procedimentos palpáveis de avaliação como provas e trabalhos que conduzem a uma nota. A \textbf{avaliação informal} se constitui de juízos de valor invisíveis que influenciam os resultados das avaliações finais e são constituídos nas interações diárias que criam representações de uns sobre os outros.~\cite{de2003ciclos} 
\end{quote}%

Sobre a avaliação formal, pode-se dizer que há uma expectativa geral de que seja conduzida conscientemente e notoriamente, e que os dados produzidos por ela sejam interpretados a partir de alguma metodologia. Já sobre a avaliação informal, a maior expectativa é justamente que ela não interfira na aprendizagem a ponto de influenciar os resultados da avaliação formal, nem lhe sirva de critério.

\subsection{Quanto a composição}%

 Outro fator importante para entendermos o processo avaliativo reside nos aspectos considerados pelo docente na composição da avaliação. Bertagna~\cite{bertagna2006} acredita que esse processo sofre influências involuntárias e externas ao ambiente educacional. E afirma que o avaliador muitas vezes se apropria do poder de avaliação para controlar outros fatores. A autora ainda afirma que:

\begin{quote}
Se entendida a avaliação como um julgamento de valor, o que se constata é que esse julgamento envolve, além da aquisição de um conhecimento, valores daqueles que julgam, suas concepções de homem, escola, sociedade.~\cite{bertagna2006}
\end{quote}%

Cada um em seus estudos, tanto Freitas~\cite{de2003ciclos} quanto Bertagna~\cite{bertagna2006}, apontam para a existência de um ``tripé avaliativo'', em que cada componente representa um dos apoios nos quais se sustenta o processo de avaliação. Para Freitas~\cite{de2003ciclos}, o resultado da avaliação é constituído por esse tripé, cujos componentes se ``complementam''. 

O componente \textbf{instrucional} que, de acordo com Freitas~\cite{de2003ciclos} é aquele pelo qual se avaliam as habilidades do aluno e o entendimento que ele possui dos conteúdos avaliados. Ainda segundo o autor este componente é realizado quase integralmente no plano da avaliação formal, através de exames, trabalhos, exercícios palpáveis, dentre outros.

%Freitas~\cite{de2003ciclos}, no entanto, alerta para o risco de que esse componente, quando mal dimensionado e nivelado, possa contribuir diretamente para a reprovação indiscriminada.   

O componente \textbf{comportamental} em que, conforme Freitas~\cite{de2003ciclos}, o aluno é avaliado em relação à sua conduta na esfera escolar. Segundo o autor, a partir do que se estabelece como comportamento correto, o docente absorve o papel de punir ou recompensar o aluno. Pode ser realizado no plano formal, que é quando o aluno possui a consciência de que está sendo avaliado pelo seu comportamento. Mas ambos os autores, Freitas~\cite{de2003ciclos} e Bertagna~\cite{bertagna2006}, relacionam esse componente a conflitos e resistência por parte dos alunos, quando empregado de forma punitiva ou autoritária.

E, por fim, o componente dos \textbf{valores e atitudes}, que atua majoritariamente no plano informal, segundo Freitas~\cite{de2003ciclos}. Sofrendo influência direta dos valores e atitudes do próprio avaliador e da sociedade, abrindo margem para o arbítrio e interferindo no desfecho da avaliação formal. A exposição de alunos à situações de constrangimento como críticas e humilhações estão associadas à esse componente, segundo ambos os autores.

Entender essas componentes é desejável para a condução de uma avaliação justa e o mais livre possível de armadilhas estruturais. Neste sentido, Bertagna~\cite{bertagna2006} afirma que "mesmo tentando promover práticas pedagógicas diferenciadas, observamos que o professor está cerceado pela avaliação que realiza''.

Freitas~\cite{de2003ciclos}, por sua vez, nos convida a refletir sobre a responsabilidade que temos, como docentes, de garantir que os alunos sejam tratados o mais igualitariamente possível:

\begin{quote}
Se os professores não forem capacitados para interagir metodologicamente nesse jogo de representações que regulam a relação professor-aluno, consciente ou inconscientemente os juízos formados determinarão o investimento que fará neste ou naquele aluno.~\cite{de2003ciclos}
\end{quote}%

Diante do que foi exposto, entende-se que a decomposição dessas influências é importante para que o docente possa identificar e atuar nos desvios dos quais ele mesmo poderia vir a ser o agente. 

\subsection{Quanto a frequência}%
Ao considerar a frequência em que a avaliação deve ser aplicada, vemos que até mesmo a autoridade legislativa se preocupou em estabelecer um norte de orientação. O trecho a seguir, retirado da Lei n{$^o$} 9.394, de 1996~\cite{brasilLDB}, popularmente conhecida como \acrfull{LDB},  demonstra a preocupação em relação a esse aspecto:


\begin{quote}
Art. 24. A educação básica, nos níveis fundamental e médio, será organizada de acordo com as seguintes regras comuns: \\
{[...]} \\
V – a verificação do rendimento escolar observará os seguintes critérios:
\renewcommand{\labelenumi}{\alph{enumi})}
\begin{enumerate}
\item avaliação \textbf{contínua e cumulativa} do desempenho do aluno, com prevalência dos aspectos qualitativos sobre os quantitativos e dos resultados ao longo do período sobre os de eventuais provas finais; \textbf{(grifo nosso)}~\cite{brasilLDB}
\end{enumerate}
\end{quote}%

Através de um levantamento de conceitos sobre avaliação, Rosado~\cite{rosado@conceitos} esboça a seguinte descrição sobre o que seria o conceito de uma \textbf{avaliação contínua}:
\begin{quote}
É frequente, também, fazer-se a distinção entre diferentes formatos de avaliação no que se refere à sua frequência e regularidade no sistema avaliativo; fala-se de avaliação contínua por oposição à avaliação pontual. A avaliação contínua é vista como acompanhando o processo de ensino-aprendizagem de forma regular.~\cite{rosado@conceitos}
\end{quote}%

Rosado~\cite{rosado@conceitos} também esclarece que a \textbf{avaliação pontual} corresponde a uma avaliação isolada. E que a questão da regularidade ainda pode ser aprofundada em relação ao momento em que a avaliação acontece. O autor destaca que pode haver uma avaliação inicial para estabelecer um ponto de partida; avaliações intermediárias no intuito de recolher informações para reajustar o processo de aprendizagem; e, uma avaliação final, que deve ser entendida como uma forma de concretizar um balanço no fim do ciclo de ensino. 



%Para fins desse trabalho iremos classificar a avaliação pontual como aquela que é conduzida em um único momento, durante toda uma etapa ou conteúdo. Ou quando, mesmo havendo mais de uma, elas não se relacionam uma com a outra cumulativamente. Conforme definição dada por Rosado~\cite{rosado@conceitos}.   

%\textbf{--> CITAR o autor que definiu desta forma e que irás seguir.}

\subsection{Quanto as funções}%

A discussão envolvendo as funções da avaliação evoluíram a partir da classificação de Bloom, Hastings e Madaus~\cite{bloom1983manual}, na qual são descritos três tipos: somativa, diagnóstica e formativa. A função somativa é descrita por eles como classificatória e quantitativa:
%\textbf{--> OK (são 3 autores no mesmo livro, fica confuso no modelo cic) FAZER AS REFERÊNCIAS aos outros dois autores}
%\textbf{--> OK TENS QUE citar os autores que conceituam as funções.}
%\textbf{--> OK REVISAR OS VERBOS, pois ao longo do texto usas diferentes pessoas na conjugação.}

\begin{quote}
A avaliação somativa é uma avaliação muito geral, que serve como ponto de apoio para atribuir notas, classificar o aluno e transmitir os resultados em termos quantitativos, sendo realizada no final de um período.~\cite{bloom1983manual}
\end{quote}

Pode-se perceber que é por meio da avaliação somativa, com atribuição de uma nota ou conceito, que acaba-se determinando se o aluno será aprovado ou reprovado. Os autores ainda adicionam que:

\begin{quote}
[...] no passado, a avaliação era de natureza exclusivamente somativa, sendo realizada apenas no final da unidade, capítulo, curso, ou semestre, quando já é tarde demais para se modificar qualquer processo, pelo menos naquele grupo de alunos.~\cite{bloom1983manual}
\end{quote}

Nessa mesma linha, Luckesi~\cite{luckesi2014avaliaccao} adverte para possíveis prejuízos na aprendizagem, apontando que o resultado da avaliação somativa é baseado em uma ou mais avaliações pontuais. Associando essa função a prática de ``decoreba'' por parte dos alunos.

A função diagnóstica muitas vezes é associada como uma avaliação pontual e inicial, com o propósito de detectar o nivelamento dos alunos e tomar as medidas necessárias para que a aprendizagem ocorra, porém como vimos anteriormente, a função diagnóstica é admitida por Luckesi~\cite{luckesi2014avaliaccao} ao longo de todo o processo de ensino e aprendizagem. Embasando essa visão, Depresbiteris~\cite{depresbiteris2017diversificar} nos diz que:

\begin{quote}
A avaliação diagnóstica possibilita ao educador e educando detectarem, ao longo do processo de aprendizagem, suas falhas, seus desvios e suas dificuldades, a tempo de redirecionarem os meios, os recursos, as estratégias e os procedimentos na direção desejada.~\cite{depresbiteris2017diversificar} 
\end{quote}

Já sobre a função formativa, Bloom, Hastings e Madaus~\cite{bloom1983manual} afirmam que: 

\begin{quote}
A avaliação formativa, como o próprio nome indica, intervém durante a formação do aluno, e não quando se supõe que o processo chegou ao término. Ela indica as áreas que necessitam ser recuperadas, de forma que o ensino e o estudo subsequentes possam ser realizados de forma mais adequada e benéfica.~\cite{bloom1983manual}
\end{quote}

Pode-se perceber, pelas duas definições acima, que a avaliação diagnóstica e a formativa se contrapõem a visão da avaliação somativa. A avaliação formativa aparece constantemente associada aos aspecto da frequência contínua, da avaliação diagnóstica, ao aspecto da formalização e ao enfoque qualitativo. Nas palavras de Perrenoud~\cite{perrenoud@93}:

\begin{quote}
Toda avaliação formativa parte de uma aposta muito otimista, a de que o aluno quer aprender e tem vontade que o ajudem, por outras palavras, a de que o aluno está disposto a revelar as suas dúvidas, as suas lacunas e as suas dificuldades de compreensão das tarefas.~\cite{perrenoud@93}
\end{quote}

Na tentativa de entender como a proposta da avaliação formativa é incorporada no sistema educacional brasileiro, Grego~\cite{grego@2013} afirma que apesar de seus preceitos estarem assegurados na LDB desde 1996, tornando-a um imperativo legal, a constatação que se faz é a de que ``não se tem logrado, ao longo de todo este período, garantir a qualidade da aprendizagem dos alunos''. E, acrescenta:

\begin{quote}
Toda avaliação formativa parte igualmente da convicção, baseada em evidências de pesquisas, de que a intervenção planejada dos professores pode criar um ambiente de aprendizagem que possibilita o engajamento do aluno, necessário a uma real aprendizagem.~\cite{grego@2013}
\end{quote}

Salientando, dessa maneira, que o ator principal, quando se trata de planejar e oportunizar a avaliação para a aprendizagem, é o docente e não o aluno. Por fim, vale ainda ressaltar que os autores Bloom, Hastings e Madaus~\cite{bloom1983manual}, que iniciaram essa discussão em 1983, partiram da premissa que ``a avaliação deve ser tanto formativa quanto somativa''.

\subsection{Quanto ao objeto avaliado}%

A avaliação também pode ser classificada em relação ao objeto avaliado. Como exemplo podemos citar a avaliação curricular, a avaliação docente, a avaliação institucional, dentre outras. Optou-se por não estender esses conceitos aqui, em razão das delimitações advindas da própria modalidade do estudo, cujo foco foi definido como sendo a partir da avaliação que o docente realiza a respeito da aprendizagem do aluno. Porém, ressalta-se que essa avaliação pode sofrer influências dos aspectos gerais de todos os outros tipos de avaliações. Por exemplo, a qualidade do currículo, o nível de proficiência do docente, a organização da instituição e outros. Todos esses objetos de avaliação podem influenciar diretamente no processo de aprendizagem.

%\textbf{--> CUIDAR porque aqui é o referencial teórico. Não podes falar da AVA, pois está no capítulo seguinte. REcomendo retirar o parágrafo a seguir. }
Não obstante, levando-se em consideração o escopo desse trabalho, os seguintes objetos de avaliação serão considerados: a avaliação do \textbf{aluno}; da \textbf{disciplina}; e a da \textbf{equipe} formada pelos profissionais responsáveis pela atividade de docência.

\subsection{Quanto ao espaço}%

O espaço pedagógico de atuação também é motivo de avaliação específica, desde a sala de aula até a instituição de ensino como um todo, já que também acredita-se que esses podem influenciar a aprendizagem. Há casos em que o processo de ensino e aprendizagem acontece em múltiplos espaços, podendo inclusive ser composto de espaço presencial combinado com espaço virtual. É o caso da modalidade semipresencial que, segundo a Portaria N{$^o$} 4.059, de 10 de dezembro de 2004, é definida como:

\begin{quote}
Art. 1{$^o$}. As instituições de ensino superior poderão introduzir, na organização
pedagógica e curricular de seus cursos superiores reconhecidos, a oferta de disciplinas
integrantes do currículo que utilizem modalidade semi-presencial, com base no art. 81
da Lei n. 9.394, de 1.996, e no disposto nesta Portaria.  \\
§ 1. Para fins desta Portaria, caracteriza-se a modalidade semi-presencial como quaisquer atividades didáticas, módulos ou unidades de ensino-aprendizagem centrados na auto-aprendizagem e com a mediação de recursos didáticos organizados em diferentes suportes de informação que utilizem \textbf{tecnologias de comunicação remota}. \\
{[...]}  \\
Art. 2{$^o$}. A oferta das disciplinas previstas no artigo anterior deverá incluir métodos e práticas de ensino-aprendizagem que incorporem o uso integrado de tecnologias de informação e comunicação para a realização dos objetivos pedagógicos, bem como prever \textbf{encontros presenciais} e atividades de tutoria. \textbf{(grifo nosso)}~\cite{brasilMEC}
\end{quote}%


%\textbf{--> CITAR um autor que dê sustentação à afirmação}

No presente estudo, foi realizado o um levantamento de informações referente ao espaço virtual de atuação que, neste trabalho, se resume a plataforma do AVA e a aplicativos disponibilizados internamente nessas plataformas. 

\subsection{Outros aspectos de classificação}%
É importante informar que na bibliografia foram identificados outros tipos de classificação para a avaliação, além daqueles que foram apresentados nesse trabalho. Entretanto, eles foram descartados por falta de embasamento teórico e/ou por não se aplicarem ao escopo do presente estudo. São eles:

Quanto ao procedimento: identificou-se a diferenciação entre avaliação estática e avaliação dinâmica, porém os conceitos eram vagos e a origem da classificação não estava clara o suficiente. Optou-se por desconsiderar essa classificação.

%\textbf{--> MANTER O PADRÃO de escrita do primeiro parágrafo. "Quanto a..." - isso facilita o entendimento do conteúdo.}
Quanto a função: mediadora, descrita como uma avaliação que propõe uma interação maior entre educador e educando, com características de uma avaliação formativa estendida. Particularidade que motivou desconsiderá-la, no intuito de não gerar confusões de conceitos. Além disso, considerando os principais autores utilizados nesse levantamento, não foram encontrados subsídios suficientes para que fosse incluída. 

Com isso, o trabalho de identificação dos tipos de avaliação chega ao fim. 

\section{Organograma dos tipos de avaliação identificadas}% 

%\textbf{--> PRECISAS FAZER uma introdução mais detalhada deste esquema e dizer de onde saíram estas divisões. Explicar claramente como essa imagem foi originada e o que ela representa. Ainda, não podes terminar uma seção com imagem, nem com itens numerados, nem com uma citação. Tens que concluir as seções com um texto seu. É importante fazer um link entre cada seção e a seguinte dela. Assim o leitor entende a lógica da organização do seu texto. Estou sentindo falta dos links entre as seções do texto até aqui.}

Para facilitar a identificação visual dos diversos tipos de avaliações encontradas, um organograma foi esquematizado da seguinte forma: na base foi adicionada a avaliação, objeto de classificação dessa seção; no primeiro nível ficaram os aspectos pelos quais a avaliação foi classificada; e, por fim, no segundo nível aparecem os tipos de avaliação que foram identificados para cada aspecto. A \refFig{fig:organogfunc} apresenta organograma completo.

\makeatletter
\setlength{\@fptop}{0pt}
\makeatother
\begin{figure}[ht]
    \centering
\begin{tikzpicture}[%
    grow=right,
    anchor=west,
    growth parent anchor=east,
    parent anchor=east,
    level 1/.style={sibling distance=2.5cm},
    level 2/.style={sibling distance=2em},
    level distance=1.5cm]

\node[root] (root) {Avaliação}
    child {node[onode] (c1) {Espaço}
        child {node[tnode] (c11) {app de terceiros}}    
        child {node[tnode] (c12) {plataforma AVA}}
    }
    child {node[onode] (c2) {Objeto}
        child {node[tnode] (c21) {equipe}}    
        child {node[tnode] (c22) {disciplina}}
        child {node[tnode] (c23) {aluno}}
    }
    child {node[onode] (c3) {Função}
        child {node[tnode] (c31) {formativa}}
        child {node[tnode] (c32) {diagnóstica}}
        child {node[tnode] (c33) {somativa}}
    }
    child {node[onode] (c4) {Frequência}
        child {node[tnode] (c41) {contínua}}
        child {node[tnode] (c42) {pontual}}
    }
    child {node[onode] (c5) {Composição}
        child {node[tnode] (c51) {valores e atitudes}}
        child {node[tnode] (c52) {comportamental}}
        child {node[tnode] (c53) {instrucional}}
    }
    child {node[onode] (c4) {Formalização}
        child {node[tnode] (c41) {informal}}
        child {node[tnode] (c42) {formal}}         
};
\end{tikzpicture}
    \caption{Organograma dos aspectos e tipos de avaliação}
    \label{fig:organogfunc}
    \vspace{2mm}
    \source{a autora}    
\end{figure}

%--> ESTE TEXTO torna-se desnecessário, pois está claro que a imagem finaliza a seção... TENS QUE fazer um texto que finalize a ideia da seção... RECOMENDO refazer o texto e retirar este que está abaixo...

O conteúdo desse capítulo é o resultado de uma pesquisa bibliográfica conduzida para esclarecer o entendimento do tema ``avaliação'' sob o enfoque de seus diversos aspectos de aplicação. Esperava-se que esses aspectos pudessem fornecer informações relevantes e estruturadas que atendessem ao propósito de classificar a avaliação e, também, como fonte de orientação teórica desse estudo. 

Ao final, pode-se perceber com certa satisfação que diversos conceitos e características emergiram desses aspectos, consolidando o alicerce pedagógico desse trabalho. Concluindo o objetivo de servir de subsídio para as próximas etapas. No próximo capítulo será discutido o conceito de um AVA.
%%%%%%%%%%%%%%%%%%%%%%%%%%%%%%%%%%%%%%%%%%%%%%%%%%%%%%%%%%%%%%%%%%%%%%%%%%%%%%%%
%%%%%%%%%%%%%%%%%%%%%%%%%%%%%%%%%%%%%%%%%%%%%%%%%%%%%%%%%%%%%%%%%%%%%%%%%%%%%%%%
%%%%%%%%%%%%%%%%%%%%%%%%%%%%%%%%%%%%%%%%%%%%%%%%%%%%%%%%%%%%%%%%%%%%%%%%%%%%%%%%
