
Embora tenha-se utilizado a matriz de mapeamento das ferramentas com os tipos de avaliação (\refTab{tab:matrizYxW}) para a identificação dos tipos de avaliação durante a análise documental, e para investigar as associações dessas duas variáveis nas propostas de avaliação para AVA, sugere-se aqui que seu uso poderia ser mais proveitoso no momento em que acontece o planejamento de uma aula ou de um curso nesses ambientes. Já que é nesse momento que um docente decide quais serão seus objetivos de avaliação.

Entender quais ferramentes que, se habilitadas para o perfil do aluno, irão possibilitar o uso das ferramentas de acompanhamento e auxílio da avaliação pode ser um exercício vantajoso ao se planejar o alcance desses objetivos. 

Se considerarmos, ainda, que o conhecimento das ferramentas é uma condição necessária para que um docente as utilize no planejamento e na execução do ensino, podemos pressupor que uma divulgação ampla do catálogo de ferramentas de um AVA, acompanhada de opções de treinamento e exemplos de utilização, poderia promover uma maior adoção delas. Presumindo-se que uma maior adoção possa influenciar de forma positiva o ensino e a aprendizagem e, talvez, potencializar a avaliação dentro da plataforma.

Como exemplo, adiciona-se aqui a experiência vivida pelo autor do documento T2 ~\cite{t2@ead} que, ao relatar o uso das ferramentas do aluno e do professor, se preocupou em descrever também as ações tomadas a partir da informação capturada com elas, demonstrando como o uso da avaliação nesses ambientes pode ser utilizado para intervir nos desvios e reorientar para a aprendizagem. 

Não obstante, espera-se que o levantamento dos aspectos da avaliação e seus tipos possam esclarecer questões sobre a prática da avaliação. Que o catálogo de ferramentas, possa auxiliar no entendimento do funcionamento de um AVA. E que o mapeamento das ferramentas com os tipos de avaliação possa ser útil na escolha dessas ferramentas de alguma forma. 

\section{Sugestões de trabalhos futuros}%
Uma outra forma de testar as informações levantadas aqui seria uma pesquisa direta com docentes e discentes de uma determinada instituição. Pode-se investigar quais ferramentas os discentes tiveram mais contato e quais ferramentas os docentes conhecem do catálogo, por exemplo. No intuito de investigar qual é o real percentual de adoção das ferramentas em relação ao total disponível. E, quem sabe, investigar o porquê algumas ferramentas são negligenciadas.

Além disso, nesse estudo foram analisadas quais ferramentas são mais utilizadas na amostragem documental, pelos docentes. Porém, nada se falou sobre a preferência dos educandos. Quais ferramentas eles escolheriam para serem avaliados? Parece uma questão interessante de se analisar quando se deseja um maior engajamento deles.
