\label{chap:map}
A partir do levantamento dos pressupostos pedagógicos das diferentes propostas de avaliação, realizado no \refCap{chap:ref} e da catalogação das ferramentas de uso do professor, apresentadas no \refCap{chap:ava}, propõem-se aqui uma estratégia de mapeamento das ferramentas com os diferentes tipos de avaliação. Para cada ferramenta do professor descrita no catálogo, pode-se identificar as características descritas de cada tipo de avaliação e, posteriormente, estruturar uma matriz com este mapeamento. A \refFig{fig:passos} ilustra o processo de mapeamento proposto.
\\
\begin{figure}[h]
    \centering
\tikzset{
    mynode/.style={
        draw, rectangle, align=center, text width=5cm, font=\small, inner sep=1ex},
    mylabel/.style={
        draw, rectangle, align=center, rounded corners, font=\small\bf, inner sep=1ex, 
        fill=cyan!30, minimum height=3cm},
    arrow/.style={
        very thick,->,>=stealth}
}

\begin{tikzpicture}[
    node distance=1.5cm,
    start chain=1 going below,
    every join/.style=arrow,
    ]
    % the chain in the center going below
    \coordinate[on chain=1] (tc);
    \node[mynode, on chain=1] (n2)
        {Estabelecimento da relação\\entre (Y)x(W)};
    \node[mynode, join, on chain=1] (n3)
        {Estruturação do mapeamento em uma matriz de associação};

    % the nodes at the top  
    \node[mynode, left=1cm of tc, anchor=south east] (n1l)
        {Levantamento dos tipos\\ de avaliação (Y)};
    \node[mynode, right=1cm of tc, anchor=south west] (n1r) 
        {Catalogação das ferramentas avaliativas (W)};

    \coordinate (n2nl) at ([xshift=-2cm]n2.north);
    \coordinate (n2nr) at ([xshift= 2cm]n2.north);
    \draw[arrow] (n1l.south -| n2nl) -- (n2nl);
    \draw[arrow] (n1r.south -| n2nr) -- (n2nr);

    % the labels on the left
    \begin{scope}[start chain=going below, xshift=-8cm, node distance=.8cm]
        \node[mylabel, on chain] {\rotatebox{90}{Identificação}};
        \node[mylabel, on chain] {\rotatebox{90}{Mapeamento}};
    \end{scope}

    % the title
    \node[above=2cm of tc, font=\bf] {Processo de Mapeamento};
\end{tikzpicture}
    \caption{Passos do mapeamento.}
    \label{fig:passos}
\end{figure}

%\textbf{--> NÃO PODES terminar uma seção com imagem. Tens que finalizar com um texto que faz um link com a seção seguinte.}
A proposta de mapeamento parte da ideia simples de que as ferramentas de um AVA são, em sua maioria, tentativas de representações virtuais de situações reais que aconteceriam em sala de aula. Portanto, cada uma delas pode representar também uma possibilidade de acompanhamento e avaliação por parte do docente. A próxima seção tentará estabelecer essa relação.  
%%%%%%%%%%%%%%%%%%%%%%%%%%%%%%%%%%%%%%%%%%%%%%%%%%%
%%%%%%%%%%%%%%%%%%%%%%%%%%%%%%%%%%%%%%%%%%%%%%%%%%%
%%%%%%%%%%%%%%%%%%%%%%%%%%%%%%%%%%%%%%%%%%%%%%%%%%%
%%%%%%%%%%%%%%%%%%%%%%%%%%%%%%%%%%%%%%%%%%%%%%%%%%%

\section{Relação entre ferramentas e avaliação}

Para estabelecer a relação entre uma ferramenta (W) e um tipo de avaliação (Y), houve a necessidade de se responder à seguinte pergunta: qual a potencial aplicabilidade da ferramenta na categoria de avaliação investigada?

Cada ferramenta de uso do professor teve sua aplicabilidade inspecionada em relação a cada um dos tipos de avalição apresentados no Organograma da \refFig{fig:organogfunc}. As possíveis respostas foram parametrizadas conforme abaixo:

\begin{itemize}
    \item Aplicável, quando a definição da avaliação descreve propriamente o uso da funcionalidade da ferramenta, considerando-se suas funcionalidades. O símbolo ``\ding{108}'' foi utilizado para representar graficamente a resposta.
    \item Parcialmente aplicável, quando a definição da avaliação descreve parte da funcionalidade. O símbolo ``\ding{115}'' representa a resposta. \item Não aplicável, quando a definição da avaliação não descreve o uso da funcionalidade. O símbolo ``\ding{53}'' representa a resposta. 
\end{itemize}

Considerando os parâmetros definidos, a ferramenta ``Acompanhar questionário'', por exemplo, quando analisada sob o aspecto da formalização, foi classificada como \textbf{aplicável para a prática da avaliação formal}, porque ela atua com ferramentas formais de aprendizagem como questionários, conforme pode ser visualizado na \refFig{fig:ferramentas}. E, também, como \textbf{parcialmente aplicável} na avaliação informal, já que fornece dados estatísticos capazes de fornecer uma avaliação informal ao docente sobre o uso do questionário pelos alunos. A Tabela \refTab{tab:relYxW} utiliza a representação gráfica mencionado anteriormente para demonstrar a Relação (Y)x(W) da ferramenta ``Acompanhar questionário''.

\begin{table}[ht!]
\setlength{\bigstrutjot}{3pt}
\settowidth\rotheadsize{long text}
\caption{Relação (Y)x(W)}
\label{tab:relYxW}
\centering
\begin{tabular}{|l|c|c|}
\addlinespace \hline
    \bigstrut \textbf{Ferramentas}  & \multicolumn{2}{c|}{Formalização}\\
\cline{2-3}
    \bigstrut
    \textbf{de avaliação}  & Formal & Informal \\
\hline
    \bigstrut[t]
    Acompanhar questionários & \ding{108} & \ding{115}\\ 
\hline
\end{tabular}
\end{table}

%\textbf{--> ESTE TRECHO está bastante confuso... Recomendo reescrever. Tens que fazer referência à imagem onde estao as ferramentas 3.4}

Do mesmo modo, a aplicabilidade de cada ferramenta de professor foi verificada quanto ao aspecto da formalização. Os resultados completos para cada item da lista de ferramentas são apresentados na \refTab{tab:tabrelYxW}.

\begin{table}[ht!]
\setlength{\bigstrutjot}{3pt}
\settowidth\rotheadsize{long text}
\caption{Lista da Relação (Y)x(W)}
\label{tab:tabrelYxW}
\centering
\begin{tabular}{|l|c|c|}
\addlinespace \hline
    \bigstrut \textbf{Ferramentas}  & \multicolumn{2}{c|}{Formalização}\\
\cline{2-3}
    \bigstrut
    \textbf{de avaliação}  & Formal & Informal \\
\hline
    \bigstrut[t]
    Gerenciamento de grupo	& \ding{108} & \ding{115}\\ \hline
    Programação do curso    & \ding{108}  & \ding{53}\\ \hline
    Acompanhar anúncios     & \ding{53} & \ding{108}\\ \hline
    Acompanhar notificações & \ding{53} & \ding{108}\\ \hline
    Recuperação de conteúdo & \ding{53} & \ding{108}\\ \hline
    Map. de Competências    & \ding{53} & \ding{108}\\ \hline
    Acompanhar colaboração  & \ding{115} & \ding{108}\\ \hline
    Acompanhar dificuldades & \ding{53} & \ding{108}\\ \hline
    Acompanhar questionários & \ding{108} & \ding{115}\\ \hline
    Revisão de textos       & \ding{108}  & \ding{53}\\ \hline 
    Acompanhar tarefas      & \ding{108}  & \ding{53}\\ \hline  
    Acompanhar aulas        & \ding{53} & \ding{108}\\ \hline
    Acompanhar exames       & \ding{108}  & \ding{53}\\ \hline  
    Registrar notas         & \ding{108}  & \ding{53}\\ \hline 
    Acompanhar notas/feedback  & \ding{53} & \ding{108}\\ \hline 
    Orientação              & \ding{108}  & \ding{53}\\ \hline        
    \bigstrut[b]
    Moderação               & \ding{108}  & \ding{53}\\  
\hline
\end{tabular}
\end{table}

 A partir desta representação, pode-se verificar que o aspecto da formalização é aplicável em pelo menos um das dois subtipos. Ou seja, uma avaliação pode ser formal ou informal e, em alguns casos, formal e informal. Mas em nenhum caso a classificação foi dispensada. O que faz todo o sentido, já que cada aspecto é um ponto de definição no planejamento da avaliação. Ou seja, dado que se deseja realizar uma avaliação, é necessário se definir como, quando, de que forma ela acontecerá.

%%%%%%%%%%%%%%%%%%%%%%%%%%%%%%%%%%%%%%%%%%%%%%%%%%%
%%%%%%%%%%%%%%%%%%%%%%%%%%%%%%%%%%%%%%%%%%%%%%%%%%%
%%%%%%%%%%%%%%%%%%%%%%%%%%%%%%%%%%%%%%%%%%%%%%%%%%%
%%%%%%%%%%%%%%%%%%%%%%%%%%%%%%%%%%%%%%%%%%%%%%%%%%%

\section{Matriz de ferramentas e avaliações}%

%\textbf{-->não utilizar penúltimo, acima, abaixo, ao lado... - somente referenciar a imagem que representa. Recomendo reescrever pois está muito coloquial, quase parecendo um passo a passo informal.}
Ao se realizar a etapa anterior para cada aspecto do Organograma das funções de avaliação, representado pela \refFig{fig:organogfunc}, chega-se em um mapeamento de todas as ferramentas do professor, consideradas pelas suas características como sendo de apoio da avaliação e do acompanhamento do ensino e da aprendizagem, em relação a todos os aspectos avaliativos encontrados. A \refTab{tab:matrizYxW} exibe os resultados completos da matriz.

%\begin{landscape}

\begin{table}[ht!]
\setlength{\bigstrutjot}{4pt}
\settowidth\rotheadsize{Valores e atitudes}
\caption{Matriz relacional entre as ferramentas de avaliação e funções de avaliação.}
\label{tab:matrizYxW}
\centering
\resizebox{\textwidth}{!}{%
\begin{tabular}{|l|c|c|c|c|c|c|c|c|c|c|c|c|c|c|c|}
\addlinespace \hline
    \bigstrut &
        \multicolumn{2}{c|}{\cellcolor{blue!25}forma} & \multicolumn{3}{c|}{\cellcolor{red!25}composição} & \multicolumn{2}{c|}{\cellcolor{green!25}freq.} & \multicolumn{3}{c|}{\cellcolor{uclagold!25}função} & \multicolumn{3}{c|}{\cellcolor{purple!25}objeto} & \multicolumn{2}{c|}{\cellcolor{cyan!25}espaço}\\
        \cline{2-3} \cline{4-6} \cline{7-8} \cline{9-11} \cline{12-14} \cline{15-16}
    \bigstrut
    \textbf{Ferramentas de avaliação}  & \rothead{Formal} & \rothead{Informal} & \rothead{Instrucional} & \rothead{Comportamental} & \rothead{Valores e atitudes} & \rothead{Pontual} & \rothead{Contínua} & \rothead{Somativa} & \rothead{Diagnóstica} & \rothead{Formativa} & \rothead{Aluno} & \rothead{Disciplina} & \rothead{Equipe} & \rothead{Plataforma} & \rothead{App de terceiros}\\
\hline
    \bigstrut[t]
    
    Geren. de grupo	&\ding{108}&\ding{115}&\ding{108} &\ding{108}&\ding{53}&\ding{53}&\ding{108}&\ding{53}&\ding{115}
    &\ding{108}&\ding{108}&\ding{53}&\ding{108}&\ding{108}&\ding{53}\\ 
    \hline
    Programação do curso &\ding{108}&\ding{53}&\ding{53} &\ding{53}&\ding{53}&\ding{53}&\ding{108}&\ding{53}&\ding{53}
    &\ding{108}&\ding{53}&\ding{108}&\ding{53}&\ding{108}&\ding{53}\\ 
    \hline 
    Acompanhar anúncios &\ding{53}&\ding{108}&\ding{108} &\ding{108}&\ding{53}&\ding{108}&\ding{53}&\ding{53}&\ding{108}
    &\ding{53}&\ding{108}&\ding{53}&\ding{53}&\ding{108}&\ding{53}\\ 
    \hline
    Acomp. notificações &\ding{53}&\ding{108}&\ding{53} &\ding{108}&\ding{53}&\ding{108}&\ding{53}&\ding{53}&\ding{108}
    &\ding{53}&\ding{108}&\ding{53}&\ding{53}&\ding{108}&\ding{53}\\ 
    \hline 
    Recup. de conteúdo &\ding{53}&\ding{108}&\ding{53} &\ding{108}&\ding{53}&\ding{53}&\ding{108}&\ding{53}&\ding{108}
    &\ding{53}&\ding{108}&\ding{115}&\ding{53}&\ding{108}&\ding{115}\\ 
    \hline
    Map. de Competências &\ding{53}&\ding{108}&\ding{53} &\ding{108}&\ding{115}&\ding{53}&\ding{108}&\ding{53}&\ding{108}
    &\ding{108}&\ding{108}&\ding{53}&\ding{115}&\ding{108}&\ding{53}\\ 
    \hline 
    Acomp. colaboração  &\ding{115}&\ding{108}&\ding{115} &\ding{108}&\ding{115}&\ding{53}&\ding{108}&\ding{115}&\ding{108}
    &\ding{108}&\ding{108}&\ding{53}&\ding{108}&\ding{108}&\ding{53}\\ 
    \hline
    Acomp. dificuldades &\ding{53}&\ding{108}&\ding{108} &\ding{115}&\ding{53}&\ding{53}&\ding{108}&\ding{53}&\ding{108}
    &\ding{108}&\ding{108}&\ding{115}&\ding{53}&\ding{108}&\ding{53}\\ 
    \hline 
    Acomp. questionários &\ding{108}&\ding{115}&\ding{108} &\ding{115}&\ding{53}&\ding{53}&\ding{108}&\ding{115}&\ding{115}
    &\ding{108}&\ding{108}&\ding{115}&\ding{53}&\ding{108}&\ding{53}\\ 
    \hline
    Revisão de textos &\ding{108}&\ding{53}&\ding{108} &\ding{53}&\ding{115}&\ding{53}&\ding{108}&\ding{108}&\ding{115}
    &\ding{115}&\ding{108}&\ding{53}&\ding{53}&\ding{108}&\ding{115}\\ 
    \hline 
    Acompanhar tarefas &\ding{108}&\ding{53}&\ding{108} &\ding{115}&\ding{53}&\ding{53}&\ding{108}&\ding{108}&\ding{115}
    &\ding{108}&\ding{108}&\ding{115}&\ding{53}&\ding{108}&\ding{115}\\ 
    \hline  
    Acompanhar aulas &\ding{53}&\ding{108}&\ding{53} &\ding{108}&\ding{53}&\ding{53}&\ding{108}&\ding{53}&\ding{108}
    &\ding{53}&\ding{108}&\ding{108}&\ding{53}&\ding{108}&\ding{115}\\ 
    \hline 
    Acompanhar exames &\ding{108}&\ding{53}&\ding{108} &\ding{115}&\ding{115}&\ding{53}&\ding{108}&\ding{108}&\ding{115}
    &\ding{53}&\ding{108}&\ding{115}&\ding{53}&\ding{108}&\ding{115}\\ 
    \hline  
    Registrar notas &\ding{108}&\ding{53}&\ding{108} &\ding{53}&\ding{53}&\ding{53}&\ding{108}&\ding{108}&\ding{53}
    &\ding{53}&\ding{108}&\ding{53}&\ding{53}&\ding{108}&\ding{53}\\ 
    \hline 
    Acomp. notas/feedback &\ding{53}&\ding{108}&\ding{108} &\ding{108}&\ding{53}&\ding{53}&\ding{108}&\ding{53}&\ding{108}
    &\ding{115}&\ding{108}&\ding{53}&\ding{53}&\ding{108}&\ding{53}\\ 
    \hline 
    Orientação &\ding{108}&\ding{53}&\ding{115} &\ding{115}&\ding{53}&\ding{53}&\ding{108}&\ding{53}&\ding{115}
    &\ding{108}&\ding{108}&\ding{53}&\ding{108}&\ding{108}&\ding{53}\\ \hline        
    \bigstrut[b]
    
    Moderação &\ding{108}&\ding{53}&\ding{53} &\ding{108}&\ding{108}&\ding{108}&\ding{53}&\ding{53}&\ding{115}
    &\ding{108}&\ding{108}&\ding{53}&\ding{108}&\ding{108}&\ding{53}\\  
\hline
\end{tabular}}
    \begin{tablenotes}
      \small
      \item Legenda: \ding{108} Aplicável; \ding{115} Parcialmente aplicável; \ding{53} Não aplicável.
    \end{tablenotes}
\end{table}
\bigskip

Com isso, pode-se verificar não apenas quais ferramentes do professor são mapeadas para a prática de uma avaliação diagnóstica, defendida por Luckesi~\cite{luckesi2014avaliaccao}, mas também pode-se vincular essa avaliação com as ferramentas que de alguma forma precisariam estar disponíveis aos alunos para que esse objetivo fosse alcançado, utilizando-se conjuntamente a tabela de vínculos apresentada na \refFig{fig:ferramentas}.

Não se pode deixar de destacar aqui que o resultado do mapeamento apresentado \refTab{tab:matrizYxW} é baseado na interpretação subjetiva da autora sobre seu entendimento das definições levantadas sobre os tipos de avaliação com cada ferramenta do perfil do professor. Assim, caso um determinado docente deseja mapear uma ferramenta de uma AVA com um tipo de avaliação, sugere-se que utilize a proposta de mapeamento e não o resultado final. Pois esse não está livre de falhas ou interpretações distorcidas.

Nada impede também que se use essa estratégia de mapeamento para relacionar os tipos de função diretamente com as ferramentas disponíveis no perfil do aluno, por exemplo. Basta seguir o mesmo processo.

A estratégia de mapeamento apresentada nesse capítulo foi criada para exercitar os entendimentos adquiridos no \refCap{chap:ref} e no \refCap{chap:ava}. Para preparar uma fonte de comparação para a pesquisa documental que será realizada no próximo capítulo e, também, para auxiliar na captura e na apresentação das possíveis propostas de mapeamento entre ferramentas e tipos de avaliação que possam surgir durante a pesquisa.

%\end{landscape}

%\begin{figure}[h]
%    \centering
%        \includegraphics[width=\textwidth]{img/avaliacao_funcoes.png}
%    \caption{Relação entre ferramentas de avaliação e funções de avaliação.}
%    \label{fig:funcoes}
%\end{figure}