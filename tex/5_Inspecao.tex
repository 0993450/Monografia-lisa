\label{chap:insp}

Neste capítulo, será apresentada uma pesquisa documental exploratória, utilizada para capturar os tipos de avaliação e as ferramentas empregadas na sua prática em Ambientes Virtuais de Aprendizagem. Bem como a apresentação de algum relacionamento entre essas duas variáveis. Para isso, foi realizada uma seleção de trabalhos acadêmicos \textbf{(--> ESTES estudos não estão em forma de artigo?) } \textcolor{blue} {no SBIE são artigos de trabalhos, no repositório aparecem dissertações, monografias, etc}que pudessem servir de amostra documental. Diante desta proposta, duas fontes foram escolhidas: a primeira se refere à uma revisão sistemática sobre avaliação em AVA~\cite{Ferreira@2016}, apresentada no \acrfull{SBIE}, em 2016; e, a segunda, refere-se ao Repositório da \acrfull{UnB}, que reúne títulos da produção científica da Instituição. 

\section{A pré-seleção amostral}%
%\textbf{-->NÃO UTILIZAR apud. Podes colocar a referência do Kitchenham, que é o autor mais utilizado neste temática.}

Segundo Kitchenham~\cite{kitchenham@2007}, uma \acrfull{RSL} é utilizada para ``identificar, analisar e interpretar todas as produções científicas disponíveis e relevantes para uma questão de pesquisa em particular, ou área temática, ou fenômeno de interesse'' (tradução do autor). A metodologia em questão foi escolhida por Ferreira \emph{et al.}~\cite{Ferreira@2016} para reunir documentos sobre ``a avaliação de Ambientes Virtuais de Aprendizagem'' com a intenção de ``identificar métodos e técnicas utilizadas na avaliação''. Ainda sobre o assunto, as autoras apresentam a seguinte dificuldade:

\begin{quote}
``A maioria dos trabalhos voltados para avaliação destacam ou o processo de desenvolvimento do ambiente virtual ou a avaliação do mesmo em relação a aspectos como interface, qualidade, clareza abstendo-se de tratar sobre o real ganho no ensino.''~\cite{Ferreira@2016}    
\end{quote}

Apesar desse obstáculo, a RSL obteve um resultado satisfatório, reunindo 46 artigos (Anexo \ref{anx:rsl}), publicados entre 2010 e 2016, sobre avaliações já implementadas em AVA's. Esse resultado revelou uma amostragem de documentos que poderiam ser utilizadas para a pesquisa documental \textbf{(-->A QUE estudo de caso você se refere?)} \textcolor{blue} {alterado}. Além dos referidos documentos, uma busca no Repositório da UnB foi realizada utilizando-se as seguintes expressões:

\begin{itemize}
    \item (avaliação E educacional);
    \item (avaliação E educação);
    \item (avaliação E distância);
    \item (avaliação E aprendizagem); e
    \item (ambiente virtual).
\end{itemize}

Para cada expressão de busca, o conjunto de resultados foi adicionado na tabela de resultados de pesquisa. O desfecho, contendo todos os conjuntos de resultados encontrados, são apresentados no Apêndice \ref{apn:tabresult}. Ao todo foram encontrados 87 resultados.

\section{A seleção do espaço amostral}%

As \refTab{tab:anex1} e \refTab{tab:apen1}, que registram os resultados da pesquisa da RSL e do Repositório da UnB, apresentaram trabalhos que fugiam do âmbito da presente pesquisa, pois reuniam propostas variadas como de avaliações realizadas em ambientes de jogos ou aplicações como softwares educacionais de matemática. Outras propostas, apresentavam avaliações em que o objeto era o próprio AVA, avaliando sua usabilidade e outros aspectos objetivos.

Dessa forma, houve a necessidade de estabelecer critérios de exclusão para descartar os resultados que não se relacionassem com avaliação para as aprendizagens do aluno no espaço do AVA. Os mesmos critérios de exclusão foram aplicados para as duas amostras separadamente.

\subsection{Critérios de exclusão por título}%
A partir das amostragens contendo os títulos dos trabalhos encontrados, iniciou-se uma triagem baseada nos referidos títulos. Durante essa leitura, os seguintes critérios de exclusão foram aplicados:
%\textbf{--> TENS QUE apresentar os critérios de exclusão. Estás colocando um texto... Reescrever os dois itens. Estava fazendo isso agora mesmo, ainda não terminei.}
\begin{itemize}
    \item Remover os títulos duplicados;
    \item Remover os títulos que apontassem explicitamente que objeto da avaliação não tenha sido o discente, a disciplina ou a equipe docente; e
    \item Remover os títulos que apontassem explicitamente que o processo de ensino e aprendizagem não aconteceu em um AVA.    
\end{itemize}

Ao final da aplicação desses critérios, sobraram 35 títulos (40\%) da amostragem inicial da pesquisa no repositório da UnB e,  seis títulos (13\%) da amostragem inicial da RSL.

\subsection{Critérios de exclusão por conteúdo} \textbf{(--> TÍTULO dá margem para interpretação dúbia. Recomendo repensar)}\textcolor{blue} {alterado} %

Ao final da triagem por títulos, iniciou-se a leitura dos resumos, objetivos e apresentações de cada um dos documentos restantes, utilizando-se os mesmos critérios de exclusão anteriores e adicionando-se mais dois critérios: \textbf{(--> NÃO ENTENDI aplicaste novamente critérios de exclusão? Por que não o fizeste na etapa anterior?)} \textcolor{blue} {tentei demonstrar o que foi feito de forma que pudesse ser reproduzido, na primeira etapa foram eliminados elementos das tabelas de resultado. na segunda, houve a busca e a leitura de cada documento.}

\begin{itemize}
    \item Remover da amostragem os documentos que não estivessem disponíveis para consulta pública;
    \item Remover as proposta cujo instrumento de avaliação não estivessem utilizando uma ou mais ferramentas de um AVA.
\end{itemize}

Com base nesses critérios, foram excluídas as propostas que utilizavam mineração de dados para análise da aprendizagem \textbf{(-->A PARTIR DE QUAL DOS CRITÉRIOS removeste este artigo?)} \textcolor{blue} {a última. talvez ficasse mais claro dizer que a avaliação não aconteceu dentro do AVA, usando as ferramenta do próprio ambiente}; avaliações cujo objetivo era analisar a plataforma ou o currículo; modelos de avaliação que utilizavam softwares específicos como uma ferramenta de terceiros usada para aplicar exames online, dentre outros; e pós-testes realizados para comparar a aprendizagem de alunos na modalidade de EAD com a modalidade presencial.

Ao final da aplicação desses critérios, sobraram cinco títulos (6\%) da amostragem inicial da pesquisa no repositório da UnB e, dois títulos (4\%) da amostragem inicial da RSL.

\subsection{Espaço amostral}%

Após a aplicação dos critérios de exclusão por título e por resumos e objetivos, os 7 títulos que restaram foram organizados na \refTab{tab:trab}. Os trabalhos identificados como T1~\cite{t1@ead} e T2~\cite{t2@ead} vieram do conjento de amostragem da revisão sistemática realizada por Ferreira \emph{et al.}~\cite{Ferreira@2017}. Os trabalhos T3~\cite{t3@ead}, T4~\cite{t4@ead}, T5~\cite{t5@ead}, T6~\cite{t6@ead}, T7~\cite{t7@ead} surgiram da pesquisa realizada no repositório da UnB.

\begin{table}[ht!]
  \footnotesize
  \centering
  \caption{Títulos selecionados para amostragem}
  \label{tab:trab}
  \begin{tabular}{|l|>{\raggedright\arraybackslash}p{9cm}|>{\raggedright\arraybackslash}p{4cm}|c|} 
    \toprule 
     T(i) & Título & Autores & Ref.  \\ 
    \midrule 
        T1 & Avaliação da Aprendizagem a partir dos Materiais Didáticos Disponíveis no Ambiente Virtual de Aprendizagem & Galdino, D. P. N.; Vale, H. C. P.; Mercado, L. P. L.; Costa, E. M. C.  &  ~\cite{t1@ead} \\ \hline 
        T2 & Utilização de Ambientes Virtuais de Aprendizagem para Ensino Médio: um estudo aplicado ao ensino de Ecologia & Carvalho Filho, Cloves Gomes & ~\cite{t2@ead}\\ \hline
        T3 & Formação a distância para conselheiros de alimentação escolar : elaboração, aplicação e avaliação & Bandeira, Luisete Moraes & ~\cite{t3@ead} \\ \hline
        T4 & O ambiente virtual de aprendizagem MOODLE como espaço multimodal de ensino de língua portuguesa & Araújo, Gizele Santos de & ~\cite{t4@ead} \\ \hline
        T5 & O uso de apresentações em slides e de um ambiente virtual de aprendizagem na perspectiva de promoção da aprendizagem significativa de conteúdos de colisões em nível de ensino médio & Lara, Anna Elisa de  & ~\cite{t5@ead} \\ \hline
        T6 & O ambiente virtual como um espaço para a autonomia na aprendizagem de línguas  & Barros, Júlia Maria Antunes & ~\cite{t6@ead} \\ \hline
        T7 & Uma proposta de ambiente virtual de aprendizagem no ensino de conceitos relacionados a equilíbrio químico  & Cardoso, Zaira Zangrando & ~\cite{t7@ead} \\
    \bottomrule 
\end{tabular}
\end{table}
\vspace{2mm}

 A amostragem final é composta por propostas variadas de EAD, que vão do Ensino Médio~\cite{t2@ead}~\cite{t4@ead}~\cite{t5@ead}~\cite{t6@ead}, passando também por cursos de línguas~\cite{t6@ead} e de formação continuada~\cite{t3@ead}, até Pós-Graduação~\cite{t1@ead}.
 
 O conjunto completo dos dados dessa tabela representa o espaço amostral da pesquisa documental que será utilizado para o estudo de caso.

\section{Questões de investigação}%

No intuito de realizar um estudo de caso documental, cujos resultados pudessem ser comparados com os levantamentos realizados, foram elaboradas questões relacionadas a cada um dos tópicos estudados. As questões de investigação são essenciais para orientar a comparação. 

A intenção primária das questões é a de apurar a adoção das ferramentas de avaliação e dos tipos de avaliação nas propostas encontradas na amostragem. A \refTab{tab:question} apresenta as questões de investigação.

\begin{table}[ht!]
\setlength{\bigstrutjot}{3pt}
\caption{Tabela de questões de investigação comparativa}
\label{tab:question}
\centering
    \begin{tabular}{|l|l|}
    \hline 
        \bigstrut
         & {Descrição das Questões}  \\
    	\hline
        \bigstrut[t]
        Q1	& Foi mencionado um dos dois AVA utilizados na listagem?\\ 	
        Q2  & As ferramentas do aluno (auxílo da aprendizagem) foram mencionadas?\\      
        Q3  & As ferramentas do docente (auxílio da avaliação) foram mencionadas?\\
        Q4  & Houve relato de alguma ferramenta não identificada no catálogo?\\          
        Q5  & Os aspectos ou os tipos de avaliação foram mencionados?\\         
        \bigstrut[b]
        Q6  & Houve associação entre ferramentas e aspectos ou tipos de avaliação?\\
    \hline
    \end{tabular}
\end{table}

\section{Coleta de dados}%

Após selecionar o espaço amostral, a leitura completa dos trabalhos foi realizada para se responder a cada uma das questões apresentadas. Os resultados observáveis de cada uma são apresentados a seguir:

\subsubsection{Questão - Q1}
Dos documentos analisados, 6 deles (T1, T3, T4, T5, T6, T7) apresentam propostas para a plataforma Moodle. O documento (T2) foi o único que apresentou uma proposta em plataforma proprietária da instituição onde o experimento foi conduzido, apresentada como Sistema Aberto de Educação (SABE).

%Dos trabalhos analisados, x deles (T1) apresentam propostas para a plataforma Moodle. Os outros, apresentaram propostas para plataformas proprietárias.

\subsubsection{Questão - Q2}
Durante a leitura dos documentos, as ferramentas utilizadas pelo perfil de uso dos alunos foram capturas utilizando-se duas marcações: ò símbolo ``-'' para as ferramentas que somente foram \textbf{mencionadas} pelos autores, ao apresentarem conceitos de AVA ou fundamentações teóricas. E o símbolo ``x'' para as ferramentas em que os autores relataram fazerem parte de sua proposta de ensino ou que relataram como tendo sido \textbf{utilizadas} pelos alunos.

Os dados de todas as observações são apresentados na \refTab{tab:Q2}.

\begin{table}[ht!]
\footnotesize
\caption{Ferramentas de uso dos discentes}
\label{tab:Q2}
\centering
\begin{tabular}{|l|c|c|c|c|c|c|c|}
\addlinespace \hline
    \bigstrut \textbf{Ferramentas}&{T1}&{T2}&{T3}&{T4}&{T5}&{T6}&{T7}\\
\hline
    \bigstrut[t]
    Bate-papo	    & 	& - & - & - &   & x & - \\ \hline
    Conferência	    &   &   & x &   &   &   &   \\ \hline	    
    Exame 		    & - &   &   &   & x &   &   \\ \hline	
    Informações	    & 	& x & x & - &   &   & x \\ \hline
    Mensagem	    & -	&   & x &   & x & x & - \\ \hline
    Notificações    & -	&   &   & - &   &   &   \\ \hline    
    Mural		    & -	&   &   &   &   &   &   \\ \hline
    Fórum		    & x	& x & x & x & x & x & x \\ \hline
    Repositório	    & x	& x & x &   & - & x & x \\ \hline
    Wiki		    & -	&   &   & - &   & - & - \\ \hline
    Glossário	    & -	&   &   &   & - &   & - \\ \hline
    Mini página	    &	& x & x &   &   &   &   \\ \hline
    Enquete		    &  	&   &   &   &   & x & x \\ \hline    
    Aula Multimídia & x	& x & x & - & x & x & - \\ \hline
    Tarefa		    & -	& x & x & x & x & x & - \\ \hline
    Questionário    &  	&   & x &   & x & x &   \\ \hline
    Painel controle&  	&   &   & x &   &   &   \\ \hline 
    Quadro de Notas &  	&   &   &   & x &   &   \\ \hline    
    \bigstrut[b]
    Suporte	    	& -	& - & - &   &   &   &  \\ 
\hline
\end{tabular}
\\Legenda: - mencionada; x utilizada.
\end{table}

\subsubsection{Questão - Q3}% 

Quanto as ferramentas de uso do perfil do docente, utilizou-se as mesmas duas marcações da questão anterior, capturando-se as que foram apenas mencionadas daqueles em que houve relato de uso. 

Em T2, apesar de a proposta ter sido planejada e executada em uma plataforma própria, o autor mencionou sete ferramentas de uso do docente e relatou ter usado ao menos cinco delas para auxílio da avaliação. Todas as ferramentas mencionadas e utilizadas faziam parte do arcabouço do catálogo levantado na Seção \ref{sec:aval} \nameref{sec:aval}.

Os dados contendo todas as observações são apresentados na \refTab{tab:Q3}.

\begin{table}[ht!]
\footnotesize
\caption{Ferramentas de uso da equipe docentes}
\label{tab:Q3}
\centering
\begin{tabular}{|l|c|c|c|c|c|c|c|}
\addlinespace \hline
    \bigstrut \textbf{Ferramentas}&{T1}&{T2}&{T3}&{T4}&{T5}&{T6}&{T7}\\
\hline
    \bigstrut[t]
    Gerenciamento de grupo	& 	& x &   & - &   &   &   \\ \hline
    Programação do curso	&  	& x & x & - & x &   & x \\ \hline
    Acompanhar notificações	& 	&   &   & - & x & x &   \\ \hline
    Recuperação de conteúdo	& 	&   &   &   & x & x &   \\ \hline
    Map. de Competências	& 	&   &   &   & x & x & x \\ \hline       
    Acompanhar colaboração	& 	& x &   & - & x & x & x \\ \hline
    Acompanhar dificuldades	& 	& - &   &   & x &   & x \\ \hline
    Acompanhar questionários& 	&   & - &   & x & x & - \\ \hline    
    Acompanhar tarefas	    & 	& x &   & - & x & x &   \\ \hline
    Acompanhar aulas	    & 	& x &   &   &   &   &   \\ \hline
    Acompanhar exame	    & 	&   &   &   & x &   &   \\ \hline    
    Acompanhar notas/feedback & - & &   &   &   &   &   \\ \hline
    Orientação              &   &   & x &   &   &   &   \\ \hline    
    \bigstrut[b]
    Moderação               & 	& x &   &   &   &   &   \\
\hline
\end{tabular}
\\Legenda: - mencionada; x utilizada.
\end{table}

\subsubsection{Questão - Q4}% 

Os documentos mencionaram também ferramentas externas integráveis com o Moodle como o blogue, uma espécie de diário virtual, e o YouTube para compartilhamento de vídeos. Software educativo para a realização de atividades específicas como simulações e animações de Física, no documento T5. Chamou a atenção, também, o documento T3 que relatou o uso de um software de apoio para a produção de apresentações guiadas e multimídia para as aulas, conhecido como \emph{Articulate Engage '09}. Dentre outros.

A tabela \refTab{tab:Q4} apresenta as ferramentas externas levantadas na pesquisa documental. A captura desses dados seguiu a mesma metodologia das ferramenentas do AVA, diferenciando as mencionadas das que foram identificadas como utilizadas.

\begin{table}[ht!]
\footnotesize
\caption{Ferramentas externas para uso do aluno ou docente}
\label{tab:Q4}
\centering
\begin{tabular}{|l|c|c|c|c|c|c|c|}
\addlinespace \hline
    \bigstrut \textbf{Ferramentas}&{T1}&{T2}&{T3}&{T4}&{T5}&{T6}&{T7}\\
\hline
    \bigstrut[t]
    Blogue	                & - & - &   &   &  & x &  \\ \hline
    Redes Sociais	        & -	& - &   &   &  &  &  \\ \hline	
    Bibliotecas Digitais	& - & - &   &   &  &  &  \\ \hline
    Software de apoio	    & 	&   & x &   &  &  &  \\ \hline
    YouTube         	    & 	&   &   & x &  &  &  \\ \hline    
    \bigstrut[b]
    software educativo      & 	&   &   &  & - &  &  \\
\hline
\end{tabular}
\\Legenda: - mencionada; x utilizada.
\end{table}

\subsubsection{Questão - Q5}% 
Durante a leitura dos documentos, observou-se que alguns tipos de avaliação foram \textbf{mencionados explicitamente} pelos autores, ou seja, pelos mesmos termos levantados no \refCap{chap:ref}. Outros foram \textbf{mencionados implicitamente}, de forma indireta e sem utilizar os termos propriamente ditos. Como, por exemplo, ao se referir ao aspecto da frequência da avaliação, os autores do documento T1 usaram o termo ``não isolada'' para se referir à avaliação contínua. Ou quando o autor do documento T2 mencionou a avaliação comportamental, pela ato de analisar os números de acessos e de mensagens trocadas entre os alunos.

Os dados contendo todas as observações são apresentados na \refTab{tab:Q5}.

%Como exemplo, o trabalho T2 mencionou a avaliação comportamental, pela avaliação de números de acessos e de mensagens trocadas. A diagnóstica, na aplicação de um questionário sobre familiarização da plataforma como pré-requisito da disciplina. E a de valores e atitudes, ao citar a necessidade de moderação sobre cópias literais sem citação de fontes. 

%Foram mencionadas explicitamente em T1 três tipos de avaliação: diagnóstica, do aluno e da plataforma. De forma indireta, sem utilizar os termos propriamente ditos, foi observado que a autora mencionou a avaliação contínua, somativa e formativa da disciplina e da equipe durante a argumentação teórica do estudo.

%No âmbito do trabalho T2, foram mencionadas de forma explícita a avaliação formativa do aluno e da disciplina. De forma indireta foi mencionada a comportamental, pela avaliação de números de acessos e de mensagens trocadas. A diagnóstica, na aplicação de um questionário sobre familiarização da plataforma como pré-requisito da disciplina. E a de valores e atitudes, ao citar a necessidade de moderação sobre cópias literais sem citação de fontes. 

\begin{table}[ht!]
\footnotesize
\caption{Tipos de avaliação mencionadas}
\label{tab:Q5}
\centering
\begin{tabular}{|l|c|c|c|c|c|c|c|}
\addlinespace \hline
    \bigstrut \textbf{Tipos}&{T1}&{T2}&{T3}&{T4}&{T5}&{T6}&{T7}\\
\hline
    \bigstrut[t]
    Formal	        & 	&   &   &   &   &   &   \\ \hline
    Informal	    &  	&   &   &   &   & x &   \\ \hline	
    Instrucional	& 	&   & - &   & - &   &   \\ \hline
    Comportamental	& 	& - &   &   & - & x & - \\ \hline
    Valores/atitudes& 	& - &   &   &   &   &   \\ \hline    
    Pontual	        & 	&   &   &   &   &   &   \\ \hline
    Contínua        & -	&   &   &   &   &   & - \\ \hline
    Somativa        & - &   & x &   &   &   &   \\ \hline    
    Diagnóstica     & x &   & x &   & - & - & - \\ \hline
    Formativa       & -	& x & x &   &   & - & x \\ \hline
    Aluno           & x	& x & x & x & x & x & x \\ \hline   
    Disciplina      & -	& x & x & x & x & x & x \\ \hline   
    Equipe          & -	&   & x & x &   &   &   \\ \hline
    Plataforma      & x	& x & x & x & x & x & x \\ \hline      
    \bigstrut[b]
    App de terceiro& 	&   &   &   &   &   &   \\
\hline
\end{tabular}
\\Legenda: - implicitamente; x explicitamente.
\end{table}

\subsubsection{Questão - Q6}% 

%Alguns documentos apresentaram algum tipo de \textbf{associação indireta} entre ferramentas e tipos de avaliação por meio dos aspectos de avaliação como frequência da avaliação, função e objeto a ser avaliado. Ou ainda, como no documento T2, em que a associação foi feita por meio de termos implícitos como a frequência do aluno (avaliação comportamental).

Alguns documentos apresentaram algum tipo de \textbf{associação direta} entre as ferramentas e os tipos de avaliação como diagnóstica, formativa, do aluno, dentre outras. Como exemplo, temos o documento T3 que associou as ferramentas de uso do aluno com os tipos de avaliação. O resultado completo da coleta de dados é apresentado na \refTab{tab:Q6}, abaixo.

\begin{table}[ht!]
\footnotesize
\caption{Associação entre ferramentas e avaliação}
\label{tab:Q6}
\centering
\begin{tabular}{|l|c|c|c|c|c|c|c|}
\addlinespace \hline
    \bigstrut \textbf{Tipos}&{T1}&{T2}&{T3}&{T4}&{T5}&{T6}&{T7}\\
\hline
    \bigstrut[t]
    formativa x fórum           & 	&   & x &  &  &  &  \\ \hline
    somativa x questionário (teste)&  	&   & x &  &  &  &  \\ \hline	
    diagnóstica x questionário	& 	&   & x &  &  &  &  \\ \hline
    formativa x questionário	& 	&   & x &  &  &  &  \\ \hline
    \bigstrut[b]    
    formativa x tarefa          & 	&   & x &  &  &  &  \\
\hline
\end{tabular}
\\Legenda: - x associação direta.
\end{table}

\section{Observações abertas}%
A leitura dos documentos possibilitou observações de questões abertas, não mapeadas na metologia e que se mostraram pertinentes no escopo dessa verificação. Mesmo não compondo os resultados analíticos, optou-se por registrá-las aqui por serem consideradas relevantes.

O documento T1, apresentou uma análise sobre a necessidade do uso de bibliotecas digitais multidisciplinares, que possam exercer em ambiente virtual a mesma função de uma biblioteca física, contendo acervos selecionados e conteúdos como livros digitais. O trabalho T2 citou a existência de uma biblioteca digital disponível para os alunos da instituição em que o estudo foi conduzido, porém a descrição dela foi próxima da descrição de um repositório de materiais didáticos. O autor T1 relatou a ambiguidade do termo biblioteca digital. 

O documento T2 utilizou um grupo de controle presencial em que o mesmo conteúdo e tarefas foram aplicados. Ao final, realizou uma avaliação sobre os resultados da verificação da aprendizagem, concluindo que houve uma resposta mais positiva no grupo que utilizou o AVA como espaço de aprendizagem. Ainda sobre o trabalho T2, o autor relatou, ao final do estudo, uma associação entre problemas de interpretação de questões com problemas na linguagem do material produzido.

O documento T2 chegou a mencionar que as ferramentas deveriam ser selecionadas de acordo com o perfil do docente e da turma, na intenção de tornar a utilização mais eficiente, porém não apresentou nenhum critério de escolha de acordo com os aspectos e tipos de avaliação.

Em T3, o autor relatou o uso de planilhas manualmente preenchidas para acompanhar questionários, discussões nos fóruns e objetivos alcançados nas tarefas.

No documento T5, a proposta foi de uma turma semipresencial, em que as aulas foram realizadas de forma presencial e as avaliação no AVA, incluindo teste síncrono que foi conduzido no laboratório da instituição.

Com isso, a verificação da adoção das ferramentas e dos aspectos avaliativos nos artigos encontrados, através do uso da amostragem da revisão sistemática da literatura, foi concluída. Uma análise dos resultados será apresentada no próximo capítulo.

%\section{Menção aos trabalhos excluídos}%