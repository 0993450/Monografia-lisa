

A \acrfull{EAD} vem se consolidando no país como modalidade de ensino em todas as etapas de escolaridade, com o auxílio de ambientes virtuais de aprendizagem (AVA). Esse cenário tem demandado novas estratégias em abordagens pedagógicas capazes de inovar e garantir o ensino e a aprendizagem nessa modalidade. Nesse contexto, percebe-se também uma preocupação com a prática de avaliação para as aprendizagens nesses ambientes, já que é pela avaliação que se espera acompanhar os resultados formais do processo de ensino e aprendizagem. Nesse sentido, o trabalho a seguir busca investigar, por meio de pesquisa documental amostral, quais os tipos de avaliação e com quais instrumentos ela está sendo conduzida nesses ambientes. Aspectos como frequência, formalização, função e objeto de avaliação serão capturados de forma quantitativa. Bem como quais funcionalidades foram as mais utilizadas. E se algum relacionamento entre esses dois enfoques está sendo considerado. Uma discussão assertiva é apresentada sobre os resultados da pesquisa documental ao final desse estudo.

%Equestões está sse \emph{resumo} passará por uma reformulação para apresentar 1) o que está sendo proposto, 2) qual o mérito da proposta, 3) como a proposta foi avaliada/validada, 4) quais as possibilidades para trabalhos futuros. 

%Por exemplo: \emph{Este trabalho apresenta informações úteis a produção de trabalhos
%científicos para descrever e exemplificar como utililzar a classe \LaTeX\ do
%Departamento de Ciência da Computação da Universidade de Brasília para gerar
%documentos. A classe \unbcic\ define um padrão de formato para textos do CIC, facilitando a
%geração de textos e permitindo que os autores foquem apenas no conteúdo. O formato
%foi aprovado pelos professores do Departamento e utilizado para gerar este documento.
%Melhorias futuras incluem manutenção contínua da classe e aprimoramento do texto
%explicativo.}